\chapter{Link Enrichment for Diffusion-based Graph Node Kernel}
\label{chap:link-enrichment}
\section{Motivation}
A powerful approach to process large heterogeneous sources of data is to use graph encodings \cite{proceeding1}, \cite{jour1} and then use graph-based learning systems. In these systems the notion of node similarity is key. A common approach is to resort to graph node kernels such as diffusion-based kernels \cite{proceeding2} where the graph node kernel measures the proximity between any pair of nodes by taking into account the paths that connect them. However, when the graph structure is affected by noise in the form of missing links, node similarities are distorted  proportionally to the sparsity of the graph and to the fraction of missing links. Two of the main reasons for this are that 1) the lower the average node degree is, the smaller the number of paths through which information can travel, and 2) missing links can end up separating a graph into multiple disconnected components. In this case, since  information cannot travel across disconnected components, the similarity of nodes belonging to different components is null. To address these problems we propose to solve a link prediction task prior to the node similarity computation and start studying the question: {\em how can we
improve node similarity using link prediction?} In this work we review both the link prediction literature and the diffusion kernel literature, select a subset of approaches in both categories that seem well suited, focus on a set of node predicting problems in the bioinformatics domain and empirically investigate the effectiveness of the combination of these approaches on the given predictive tasks. The encouraging result that we find is that all the strategies for link prediction we examined consistently enhance the performance on downstream predictive tasks, often significantly improving state of the art results.
\section{Method}
Often the relational information that defines the graph structure is incomplete because certain relations are not known at a given moment in time or have not been yet investigated. When this happens the resulting graphs tend to become sparse and composed of several disconnected components. Diffusion-based kernels are not suited in these cases and show a degraded predictive capacity. Our key idea is to introduce a {\em link enrichment phase} that can address both issues and enhance the performance of diffusion-based systems.

Given a link prediction algorithm $M$, a diffusion-based graph node kernel $K$ and a sparse graph $G=(V, E)$ in which $|V| = n$ and $|E| = m$, with $m \approx n$ the link enrichment method consists of two phases:

\begin{itemize}

\item enrichment: the link prediction algorithm $M$ is used to score all possible $\frac{n(n-1)}{2}-m$ missing links. The top scoring $t$ links are added to $E$ to obtain $E'$ that defines the new graph $G'=(V,E')$.

\item kernel computation: the diffusion-based graph node kernel $K$ is applied to graph $G^{'}$ to compute the kernel matrix $K'$ which captures the similarities between any couple of nodes, possibly belonging to different components in the graph $G$. 
\end{itemize}

The kernel matrix $K'$ can be used directly by a kernelized learning algorithm, such as a support vector machine, to make predictive inferences.
\section{Experiments}
\label{evaluation} 

To empirically study the answer to the question: {\em how can we improve node
similarity using link prediction?} we would need to define a taxonomy of
prediction problems on graphs that make use of the notion of node similarity
and analyze which link prediction strategies can be effectively coupled with
specific node similarity computation techniques for each given class of
problems. In addition we should also study the quantitative relation between
the degree of missingness and the size of the improvement offered by
prepending the link prediction to the node similarity assessment. In this
paper we start such endeavor restricting the type of predictive problems to
that of node ranking in the sub-domain of gene-disease association studies
with a fixed but unknown degree of missingness given by the current medical
knowledge. More in detail, the task, known as {\em gene prioritization},
consists in ranking candidate genes based on their probabilities to be related
to a disease on the basis of a given a set of genes experimentally known to be
associated to the disease of interest. We have studied the proposed approach on the following 4 datasets:

\textbf{BioGPS:} a gene co-expression graph (7311 nodes and 911294 edges) 
constructed from the BioGPS dataset, which contains 79 tissues, measured with
the Affymetrix U133A array. Edges are inserted when the pairwise Pearson
correlation coefficient (PCC) between genes is larger than 0.5.

\textbf{HPRD:} a database of curated proteomic information pertaining to
human proteins. It is derived from \cite{jour6} with 9,465 vertices and 37,039
edges. We employ the HPRD version used in \cite{jour5} that contains 7311 nodes and 30503 edges. 

\textbf{Phenotype similarity:} we use the OMIM \cite{jour4} dataset and the
phenotype similarity notion introduced by Van Driel et al. \cite{jour7} based on the relevance and the frequency of the Medical Subject Headings (MeSH) vocabulary terms in OMIM documents. We built the graph linking those
genes whose associated phenotypes have a maximal phenotypic similarity greater
than a fixed cut-off value. Following \cite{jour7}, we set the
similarity cut-off to $0.3$. The resulting graph has 3393 nodes and 144739 edges.

\textbf{Biogridphys:} this dataset encodes known physical interactions
among proteins. The idea is that mutations can affect physical interactions by
changing the shape of proteins and their effect can propagate through protein
graphs. We introduce a link between two genes if their products interact. The resulting graph has 15389 nodes and 155333 edges.

\subsection{Evaluation Method}

To evaluate the performance of the diffusion kernels, we follow \cite{proceeding3}: we choose $14$ diseases with at least $30$ confirmed genes. For each disease, we construct a positive set $\mathcal{P}$ with all
confirmed disease genes. To build the negative set $\mathcal{N}$, we randomly sample a set of genes that are associated at least to one disease class, but not related to the class which defines the positive set such that $\vert \mathcal{N} \vert = \frac{1}{2} \vert \mathcal{P} \vert$. We replicate this procedure 5 times\footnote{Note that the positive set is held constant, while the negative set varies.}. We assess the performance of kernels via a 3-fold CV, where, after partitioning  the dataset $\mathcal{P} \cup \mathcal{U}$ in 3 folds, we use one fold for training model using SVM and the two remaining folds for testing. For each test gene $g_i$, the model returns a score $s_i$ proportional to the likelihood of being associated to the disease. Next a decision score $q_i$ is computed as the top percentage value of $s_i$ among all candidate gene scores.
We collect all decision scores for every test genes to compute the area under the curve for the receiver operating characteristic (AUC-ROC).  The final performance on the disease class is obtained by taking average over 3 folds $\times$ 5 trials.


\textbf{Model Selection}: The hyper-parameters of the various methods are set using a 3-fold on training set in which one fold is used for training the model and two remaining folds are used for validation. We try the values for LEDK and MEDK in $\lbrace  0.01, 0.05, 0.1 \rbrace$, time steps in MDK in $\lbrace 3, 5, 10 \rbrace$ and RLK parameter in $\lbrace 0.01, 0.1, 1 \rbrace$. For CDNK, we try for the degree threshold value in $\lbrace 10,\ 15,\ 20 \rbrace$, clique size threshold in $\lbrace 4,\ 5 \rbrace$, maximum radius in $\lbrace 1,\ 2 \rbrace$, maximum distance in $\lbrace 2,\ 3,\ 4 \rbrace$. The number of links used for the enrichment are chosen in $\lbrace 40\%,\ 50\%,\ 60\%,\ 70\% \rbrace$ of the number of existing links. Finally, the regularization tradeoff $C$ for the SVM is chosen in $\lbrace 10^{-4}, 10^{-3}, 10^{-2},\ 10^{-1}, 1,\ 10,\ 10^2, \\ 10^3,\ 10^4 \rbrace$.
\section{Results and discussion}
\definecolor{cadetgrey}{rgb}{0.8721875,0.8721875,0.8721875}
\newcolumntype{g}{>{\columncolor{cadetgrey}}c}

{\setlength{\extrarowheight}{2pt}
\begin{table*}[!htbp]
\vspace*{-0.1cm}
\centering
\caption{\textit {Predictive performance on 14 gene-disease associations using four different graphs induced by the BioGPS, Biogridphys, Hprd and Omim. We report the average AUC-ROC (\%) and standard deviations for all difussion-based kernels with (+) and without (-) link enrichment.}}
\label{table:results1}
\setlength{\tabcolsep}{0.6mm}
\begin{tabular}{|c|c|g|c|g|c|g|c|g|}
\hline
 & \multicolumn{2}{c|}{\textbf{BioGPS}} & \multicolumn{2}{c|}{\textbf{Biogridphys}} & \multicolumn{2}{c|}{\textbf{Hprd}} & \multicolumn{2}{c|}{\textbf{Omim}}\\
 \hline
Disease & - & + & - & + & - & + & - & + \\
\hline
1 & 60.3$\pm$1.5 & 63.4$\pm$1.0 & 73.1$\pm$4.1 & 77.1$\pm$2.9 & 75.5$\pm$0.2 & 77.5$\pm$0.9 & 85.3$\pm$1.1 & 86.9$\pm$1.5 \\
2 & 53.7$\pm$1.4 & 63.4$\pm$3.8 & 56.6$\pm$3.4 & 61.3$\pm$4.1 & 57.1$\pm$0.9 & 60.2$\pm$1.8 & 75.0$\pm$2.2 & 76.5$\pm$2.4 \\
3 & 50.2$\pm$0.4 & 58.6$\pm$3.0 & 58.9$\pm$5.9 & 67.5$\pm$7.7 & 61.8$\pm$3.6 & 70.7$\pm$3.8 & 77.3$\pm$1.8 & 83.1$\pm$0.9 \\
4 & 61.5$\pm$0.9 & 72.2$\pm$2.2 & 65.7$\pm$4.1 & 74.6$\pm$4.2 & 67.3$\pm$1.1 & 71.9$\pm$2.2 & 90.2$\pm$1.2 & 92.1$\pm$1.2 \\
5 & 55.1$\pm$0.4 & 61.7$\pm$0.9 & 54.2$\pm$4.8 & 60.7$\pm$4.0 & 57.7$\pm$1.6 & 67.0$\pm$1.8 & 76.4$\pm$0.8 & 81.9$\pm$1.5 \\
6 & 60.8$\pm$0.9 & 67.9$\pm$2.2 & 60.6$\pm$3.6 & 65.9$\pm$3.5 & 66.8$\pm$1.3 & 71.9$\pm$2.3 & 79.9$\pm$2.4 & 83.3$\pm$1.2 \\
7 & 68.1$\pm$1.4 & 73.4$\pm$0.7 & 57.7$\pm$3.2 & 63.7$\pm$4.0 & 68.9$\pm$2.1 & 72.5$\pm$1.2 & 81.0$\pm$1.2 & 84.1$\pm$1.0 \\
8 & 69.2$\pm$2.3 & 74.0$\pm$2.2 & 68.1$\pm$3.6 & 72.6$\pm$2.5 & 76.6$\pm$2.2 & 80.3$\pm$2.8 & 85.4$\pm$2.2 & 91.0$\pm$1.0 \\
9 & 62.0$\pm$1.6 & 64.5$\pm$1.4 & 68.7$\pm$4.6 & 71.7$\pm$4.3 & 68.4$\pm$2.5 & 75.0$\pm$3.2 & 78.5$\pm$0.2 & 80.6$\pm$0.6 \\
10 & 67.5$\pm$2.9 & 72.9$\pm$1.8 & 58.8$\pm$3.2 & 66.1$\pm$3.8 & 65.8$\pm$3.4 & 74.4$\pm$2.6 & 86.1$\pm$0.6 & 87.8$\pm$0.3 \\
11 & 58.7$\pm$1.8 & 62.3$\pm$1.5 & 58.2$\pm$1.2 & 61.6$\pm$1.7 & 60.1$\pm$1.1 & 64.2$\pm$1.5 & 82.0$\pm$1.4 & 83.6$\pm$0.9 \\
12 & 64.0$\pm$1.3 & 73.6$\pm$1.7 & 59.3$\pm$2.1 & 67.0$\pm$2.8 & 60.8$\pm$1.1 & 68.8$\pm$2.8 & 82.0$\pm$1.8 & 85.9$\pm$1.7 \\
13 & 56.5$\pm$0.9 & 63.3$\pm$2.4 & 55.8$\pm$1.1 & 65.1$\pm$4.2 & 66.4$\pm$1.3 & 71.8$\pm$1.7 & 83.1$\pm$2.8 & 87.5$\pm$2.5 \\
14 & 55.2$\pm$0.3 & 62.3$\pm$1.2 & 55.6$\pm$1.6 & 63.5$\pm$4.0 & 66.3$\pm$2.3 & 71.1$\pm$2.8 & 97.4$\pm$0.1 & 99.0$\pm$0.4 \\
\hline
$\overline{AUC}$ & 60.2$\pm$0.3 & 66.7$\pm$1.2 & 60.8$\pm$1.6 & 67.0$\pm$4.0 & 65.7$\pm$2.3 & 71.2$\pm$2.8 & 82.8$\pm$0.1 & 86.0$\pm$0.4 \\
\hline
\end{tabular}
\end{table*}

\section{Results and Discussion}
\label{results-discussion}
In Table \ref{table:results1} we report a synthesis of all the experiments. Each row represent a different disease, in the columns we consider the different sources of information used to build the underlying graph (BioGPS, Biogridphys, Hprd, Omim). Note that each resource yields a graph with different characteristic sparsity and number of components. We compare the average AUC-ROC scores in two cases: plain diffusion kernel (denoted by a ``-'' symbol) and diffusion kernel on a modified graph $G'$ (denoted by a ``+'' symbol) which includes the novel edges identified by a link prediction system. Here we report the aggregated results (a detailed breakdown is available in \textit{Appendix}\footnote{https://github.com/dinhinfotech/ICANN/blob/master/appendix.pdf}) where we have averaged not only across a random choice of negative genes, but also among the type of diffusion kernel and the type of link prediction. The noteworthy result is how consistent the result is: each link prediction method improves each diffusion kernel algorithm, and on
average using link prediction yields a 15\% to 20\% relative error reduction for diffusion-based methods. What varies is the amount of improvement, which depends on the coupling between the four elements: the disease, the
information source, the link prediction method and the diffusion kernel algorithm. In specific we obtain that the largest improvement is obtained for disease 3 (connective) where we have a maximum improvement of 20\% ROC points, while the minimum improvement is for disease 8 (immunological) with a minimal improvement of 0\% ROC points (see in the detailed report). On average the largest improvement is of 13\% ROC points, while the smallest improvement is on average of 1\% ROC point. Such stable results are of interest since diffusion-based methods are currently state-of-the-art for gene-disease prioritization tasks, and hence a technique that can offer a consistent and relatively large improvement can have important practical consequences in the understanding of disease mechanisms.
\section{Conclusion}
In this paper we have proposed the notion of {\em link enrichment} for diffusion kernels, that is, the idea of carrying out the computation of information diffusion on a graph that contains edges identified by link
prediction approaches. We have discovered a surprisingly robust signal that indicates that diffusion-based node kernels consistently benefit from the coupling with similarity-based link prediction techniques on large
scale datasets in biological domains.

In future work we will carry out a more fine grained analysis, defining a taxonomy of prediction problems on graphs that make use of the notion of node similarity and analyze which link prediction strategies can be
effectively coupled with specific node similarity computation techniques for a given problem class. In addition we will study the quantitative relation between the degree of missingness and the size of the improvement
offered by prepending the link prediction to the node similarity assessment. Finally, we will extend the analysis to the more complex case of kernel integration and data fusion, i.e. when multiple heterogeneous information sources are used jointly to define the predictive task.