\chapter*{Abstract}
We have been witnessing the explosive growth of biological data in terms of volume and variety, thanks to the development of technologies related to web services and embeded systems. Such data can naturally be represented as graphs. It is common that entities are encoded in different data sources. On the one hand, this provides a great opportunity for us to have unified views about entities of interest. On the other hand, this poses challenges for scientists to wisely extract knowledge from such huge amount of data which normally cannot be done without the help of automated learning systems. Therefore, there is a need of developing smart learning systems which support experts to form and assess hypotheses in biology and medicine. The problem of data integration in general or graph-based data integration in specific needs to be efficiently solved if we desire to build high performance learning systems. The contributions of this thesis focus on addressing challenges for graph-based data integration problem with the aim of achieving high performance systems: \textit{node similarity measure}, \textit{graph sparsity}, \textit{scalability}, \textit{effectiveness and efficiency}.\\

The first contribution is the definition of an efficient graph node kernel named Conjunctive disjunctive graph node kernel which aims at defining the similarity between any graph node couple. The kernel first decomposes the input graph into a collection of connected sparse graphs. It then develops a suitable kernel that explicitly models the configuration of each node's context to measure node proximity. The proposed kernel shows state of the art performance for graph node kernels.\\

The second contribution of this thesis aims to deal with graph sparsity problem by introducing a link prediction method whose objective is to enrich links of a graph. In this method we first represent each link connecting two nodes by a graph composed of their neighborhood subgraphs. We then cast the link prediction problem as a binary classification task over obtained graphs in which we employ an efficient decompositional graph kernel for graph similarity. Empirical evaluation proves the promissing of the method.\\

The third contribution is a proposed method to enrich the performance of diffusion-based graph node kernels when working with sparse graphs caused by the lack of information. By adding link enrichment phase before employing diffusion-based kernels, we empirically show a robust and large effect for combination of a number of link prediction and a number of diffusion-based kernels on several real gene-disease association problems.\\

The fourth contribution copes with the scalability problem by proposing scalable kernel-based gene prioritization method (Scuba). Scuba is optimized to deal with strongly unbalanced setting and is able to deal with both large amount of candidate genes and arbitrary number of data sources. It outperforms and enhances the scalability, efficacy comparing with existing methods for disease gene prioritization.\\

The last contribution is an approach for graph integration targeting to solve disease gene prioritization problem. In this method, the common genes between graph layers, derived from biological sources, are connected by disjunctive links. Then a particular graph node kernel is adopted to exploit topological graph features from all layers for measuring gene similarities. The state of the art performance on different experimental settings confirms the strength of the method. 
